% Paper template for TAR 2020
% (C) 2014 Jan Šnajder, Goran Glavaš, Domagoj Alagić, Mladen Karan
% TakeLab, FER

\documentclass[10pt, a4paper]{article}

\usepackage{tar2020}

\usepackage[utf8]{inputenc}
\usepackage[pdftex]{graphicx}
\usepackage{booktabs}
\usepackage{amsmath}
\usepackage{amssymb}

\title{
Search less, research more!\\
Improving information retrieval in the face of covid-19\\
}

\name{Mario Šaško, Ivan Lovrenčić, Nikola Buhiniček}

\address{
University of Zagreb, Faculty of Electrical Engineering and Computing\\
Unska 3, 10000 Zagreb, Croatia\\
\texttt{\{mario.sasko, ivan.lovrencic, nikola.buhinicek\}@fer.hr}\\
}


\abstract{
As the persistent COVID-19 outbreak transforms the world, scientists are vigorously researching the virus to find a cure. For researchers to stay updated with the latest developments, they have to find the time to evaluate current advancements and potentially apply them in their research. However, because of the global initiative to solve this crisis, new research is getting published continually. That causes an abundance of research papers, where most of them do not deliver any significance to the on-going cure development. To help researchers find the most prominent improvements, we propose the solution in the form of an information retrieval model, that we adapted to the scientific domain of this problem. Our model will allow researchers to find the most relevant papers for each of their COVID-19 related queries, and in the process, save some valuable time.
}

\begin{document}

\maketitleabstract

\section{Introduction}

In the last days of 2019, numerous reports about the previously-unknown virus have appeared in Wuhan, China. Not even a few weeks after that, the World Health Organization (WHO) started alerting about a seeming global health crisis. Six months later, the highly-infectious novel Sars-Cov-2 virus has infected more than five million people and more than 340,000 people have died. Furthermore, researchers believe that this emergency will not cease until there is a viable cure, and, naturally, as a part of the global response to find it, researchers are publishing more papers than ever. However, from that arose another challenge. There is now an abundance of research papers being published, and researchers are keen to find a way to retrieve only the most prominent ones to save some valuable time.

To help researchers, the White House and the coalition of leading research groups came up with an initiative to prepare the COVID-19 Open Research Dataset (\emph{CORD-19}\footnote{\texttt{https://bit.ly/2ZMyMTZ}}). The dataset contains over 128,000 scholarly articles about the on-going pandemics. This freely available dataset is given to the global research community in order to apply natural language processing (NLP) and other AI techniques, that could give us a better understanding of the Sars-CoV-2 virus and, sequentially, help scientists find the most relevant research about the on-going pandemic.

To contribute, we have decided to develop our IR model. The main intention of our approach was to allow researchers to obtain the most relevant papers based on their query input. To achieve that, we have used a stratified model, that through each level, filters and ranks papers based on their query relevance.  The model starts with a biomedical named entity recognition (BioNER) to filter out papers that are not related to the current query. Furthermore, the model proceeds by utilizing the word (word2vec) and document (doc2vec) level embeddings to determine the remaining papers significance rank. In the end, we are left with a list of query-related papers, which are sorted based on their relevance score.

In the following section, we will discuss the approaches of other highly-ranked proposed solutions and compare them to our own. Furthermore, in section 3., we are going to describe our proposal in detail, including each level of our final model. Moreover, in section 4., we will talk more about evaluation and results.  We had challenges in the evaluation part, as it was hard to evaluate the significance of the research paper from the domain we are not experts. Lastly, we finish with a conclusion and ideas for future work.

\section{Related work}

The mentioned CORD-19 dataset came along with a series of coronavirus related \emph{tasks}\footnote{\texttt{https://bit.ly/3gyyaHv}} and so far there has been over 1,400 submissions. There could be an article written about each of those solutions, but we will only shortly cover the approaches of few most accepted ones.

The officially accepted submission for this problem is actually using only paper abstracts for relevance determination. The query is processed into keywords which are stemmed and concatenated with corona related terms (covid, cov, -cov, hcov). Then the papers are filtered by checking if the abstract contains all of the queried words and at least one corona related term. Later on, for the remaining papers, relevance scores are calculated based on the keyword counts and the length of the abstract. Based on our work and the issues we encountered, the decision to use only abstracts for retrieval isn't justifiable. Our model is utilising more of the available data, weighting parts of papers differently and not relying just on the short summaries.

There is a different approach, also highly rated, that turned to a probabilistic IR model, BM25, using it along with document embeddings. They are using the BM25 search index, created from paper abstracts, as their base and boosting the performance with an Annoy index made out of 786 dimensional Specter Vectors, which represent document embeddings. This is just one of many submissions that is using some level of embeddings while we decided to use both word and document embeddings to get a better insight. Alongside with embeddings, another often used NLP tool is named entity recognition (NER). To adapt more to the task, we used BioNER, a specialised NER for entities commonly searched in this scientific domain.

Another acclaimed solution is more machine learning oriented. They are relying on the article body which they parse using NLP. Each document is then turned into a vector, based on TF-IDF, and the rest of work is done by applying k-means clustering. Our model though, isn't going into ML but there are plenty of ML approaches for this task which is worth mentioning.

\subsection{First subsection}
\label{sec:first}

This is a subsection of the second section.

\subsection{Second subsection}

This is the second subsection of the second section. Referencing the (sub)sections in text is performed as follows: ``in Section~\ref{sec:first} we have shown \dots''.

\subsubsection{Sub-subsection example}

This is a sub-subsection. If possible, it is better to avoid sub-subsections.

\section{Extent of the paper}

The paper should have a minimum of 3 and a maximum of 4 pages, plus an additional page for references.

\section{Figures and tables}

\subsection{Figures}

Here is an example on how to include figures in the paper. Figures are included in \LaTeX{} code immediately \textit{after} the text in which these figures are referenced. Allow \LaTeX{} to place the figure where it believes is best (usually on top of the page of at the position where you would not place the figure). Figures are referenced as follows: ``Figure~\ref{fig:figure1} shows \dots''. Use tilde (\verb.~.) to prevent separation between the word ``Figure'' and its enumeration.

\subsection{Tables}

There are two types of tables: narrow tables that fit into one column and a wide table that spreads over both columns.

\subsubsection{Narrow tables}

Table~\ref{tab:narrow-table} is an example of a narrow table. Do not use vertical lines in tables -- vertical tables have no effect and they make tables visually less attractive. We recommend using \textit{booktabs} package for nicer tables.

\begin{table}
\caption{This is the caption of the table. Table captions should be placed \textit{above} the table.}
\label{tab:narrow-table}
\begin{center}
\begin{tabular}{ll}
\toprule
Heading1 & Heading2 \\
\midrule
One & First row text \\
Two   & Second row text \\
Three   & Third row text \\
      & Fourth row text \\
\bottomrule
\end{tabular}
\end{center}
\end{table}

\subsection{Wide tables}

Table~\ref{tab:wide-table} is an example of a wide table that spreads across both columns. The same can be done for wide figures that should spread across the whole width of the page.

\begin{table*}
\caption{Wide-table caption}
\label{tab:wide-table}
\begin{center}
\begin{tabular}{llr}
\toprule
Heading1 & Heading2 & Heading3\\
\midrule
A & A very long text, longer that the width of a single column & $128$\\
B & A very long text, longer that the width of a single column & $3123$\\
C & A very long text, longer that the width of a single column & $-32$\\
\bottomrule
\end{tabular}
\end{center}
\end{table*}

\section{Math expressions and formulas}

Math expressions and formulas that appear within the sentence should be written inside the so-called \emph{inline} math environment: $2+3$, $\sqrt{16}$, $h(x)=\mathbf{1}(\theta_1 x_1 + \theta_0>0)$. Larger expressions and formulas (e.g., equations) should be written in the so-called \emph{displayed} math environment:

\[
b^{(i)}_k = \begin{cases}
1 & \text{if
    $k = \text{argmin}_j \| \mathbf{x}^{(i)} - \mathbf{\mu}_j \|,$}\\
0 & \text{otherwise}
\end{cases}
\]

Math expressions which you reference in the text should be written inside the \textit{equation} environment:

\begin{equation}\label{eq:kmeans-error}
J = \sum_{i=1}^N \sum_{k=1}^K
b^{(i)}_k \| \mathbf{x}^{(i)} - \mathbf{\mu}_k \|^2
\end{equation}

Now you can reference equation \eqref{eq:kmeans-error}. If the paragraph continues right after the formula

\begin{equation}
f(x) = x^2 + \varepsilon
\end{equation}

\noindent like this one does, use the command \emph{noindent} after the equation to remove the indentation of the row.

Multi-letter words in the math environment should be written inside the command \emph{mathit}, otherwise \LaTeX{} will insert spacing between the letters to denote the multiplication of values denoted by symbols. For example, compare
$\mathit{Consistent}(h,\mathcal{D})$ and\\
$Consistent(h,\mathcal{D})$.

If you need a math symbol, but you don't know the corresponding \LaTeX{} command that generates it, try
\emph{Detexify}.\footnote{\texttt{http://detexify.kirelabs.org/}}

\section{Referencing literature}

References to other publications should be written in brackets with the last name of the first author and the year of publication, e.g., \citep{chomsky-73}.  Multiple references are written in sequence, one after another, separated by semicolon and without whitespaces in between, e.g., \citep{chomsky-73,chave-64,feigl-58}. References are typically written at the end of the sentence and necessarily before the sentence punctuation.

If the publication is authored by more than one author, only the name of the first author is written, after which abbreviation \emph{et al.}, meaning \emph{et alia}, i.e.,~and others is written as in \citep{johnson-etc}. If the publication is authored by only two authors, then the last names of both authors are written \citep{johnson-howells}.

If the name of the author is incorporated into the text of the sentence, it should not be in the brackets (only the year should be there). E.g.,~``\citet{chomsky-73}
suggested that \dots''. The difference is whether you reference the publication or the author who wrote it.

The list of all literature references is given alphabetically at the end of the paper. The form of the reference depends on the type of the bibliographic unit: conference papers,
\citep{chave-64}, books \citep{butcher-81}, journal articles
\citep{howells-51}, doctoral dissertations \citep{croft-78}, and book chapters \citep{feigl-58}.

All of this is automatically produced when using BibTeX. Insert all the BibTeX entries into the file \texttt{tar2020.bib}, and then reference them via their symbolic names.

\section{Conclusion}

Conclusion is the last enumerated section of the paper. It should not exceed half of a column and is typically split into 2--3 paragraphs. No new information should be presented in the conclusion; this section only summarizes and concludes the paper.

\section*{Acknowledgements}

If suitable, you can include the \textit{Acknowledgements} section before inserting the literature references  in order to thank those who helped you in any way to deliver the paper, but are not co-authors of the paper.

\bibliographystyle{tar2020}
\bibliography{tar2020}

\end{document}
